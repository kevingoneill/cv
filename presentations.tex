\section{Talks}
\textbf{O'Neill, K.} (2025). Mutual relationships between modal
cognition and metacognition. Part of the symposium \emph{Bridges
between modal cognition and episodic simulation}. \emph{European
Society for Philosophy \& Psychology}, \emph{Conference on Cognitive
Computational Neuroscience}.

\textbf{O'Neill, K.}, Singmann, H., \& Fleming, S. M. (2025)
Extreme-value signal detection theory provides a less biased measure
of metacognitive efficiency. \emph{Metacognitive Science Satellite
Conference.}

\textbf{O'Neill, K.} (2025). Rethinking counterfactual and causal
sensitivity. Part of the symposium \emph{Understanding and Modeling
Causal Judgments: perspectives from philosophy, cognitive science, and
economics}. \emph{Philosophy, Politics, and Economics Society}.

\textbf{O'Neill, K.}, Lagnado, D., \& Fleming, S. M. (2025)
Metacognition as modal cognition. \emph{Association for the
Scientific Study of Consciousness}.

\textbf{O'Neill, K.}, Henne, P., Icard, T., Quillien, T., \&
De~Brigard, F. (2023). Disentangling double
prevention. \emph{Society for Philosophy and Psychology}.

\textbf{O'Neill, K.}, Krasich, K., Murray, S., Brockmole, J.,
Nuthmann, A., \& De~Brigard, F. (2023). Fixation duration variability
increases with mind wandering during scene viewing. \emph{Current
Issues in Mind-Wandering Research}.

\textbf{O'Neill, K.}, Stern, R., \& Eva, B. (2023). Colliding
intuitions about causeless correlations: an investigation of human
reasoning errors in collider causal structures. \emph{Southern
Society for Philosophy and Psychology}.

Henne, P. \& \textbf{O'Neill, K.} (2022-2023). Double prevention,
causal judgments, and counterfactuals. \emph{Invited talk for the
Causality in Cognition Lab, Stanford}; \emph{Southern Society for
Philosophy and Psychology}.

\textbf{O'Neill, K.}, Henne, P., Pearson, J., De~Brigard,
F. (2022). Measuring and modeling confidence in human causal
judgment. \emph{Cognitive Science Society}; \emph{Society for
Philosophy and Psychology}; \emph{Southern Society for Philosophy and
Psychology}.

Krasich, K., \textbf{O'Neill, K.}, De~Brigard, F. (2022). Eye
tracking mental simulations during retrospective causal
reasoning. \emph{Cognitive Science Society}; \emph{Society for
Philosophy and Psychology}; \emph{Southern Society for Philosophy and
Psychology}; \emph{London Judgment and Decision-Making Seminar}.

\textbf{O'Neill, K.} (2022). Disentangling confidence and causal
judgment. \emph{Invited talk for the Consciousness Club, Meta Lab,
University College London}.

\textbf{O'Neill, K.} (2022). Confidence \& singular causal
judgment. \emph{Invited talk for the Cognitive and Neural
Computation Lab, University of California Irvine}.

Khoudary, A., \textbf{O’Neill, K.}, Faul, L., Murray, S., Smallman,
R., De~Brigard, F. (2021-2022). Neural differences between internal
and external episodic counterfactual thoughts. \emph{Neuromatch
Conference 4.0}.

\textbf{O'Neill, K.}, Henne, P., Bello, P., Pearson, J., De~Brigard,
F. (2021). Degrading causation. \emph{Invited talk at Causal
Cognition Lab, UCL}; \emph{XPhi Europe}.

Bello, P., \textbf{O'Neill, K.}, Bridewell, W. (2019). Artificial
agency requires attention: the case of intentional action. In
\emph{AAAI Spring Symposium: Towards Conscious AI Systems}.

\textbf{O'Neill, K.}, Bridewell, W., Bello, P. (2018). Time-based
resource sharing in ARCADIA. \emph{40th Annual Meeting of the
Cognitive Science Society}.

\textbf{O’Neill, K.}, Bringsjord, S. Solving the lottery paradox in
a cognitive calculus. (2016). \emph{International Association for
Computing and Philosophy}.


\line\section{Poster Presentations}

\textbf{O'Neill, K.}, Lagnado, D., \& Fleming, S. M. (2025)
Metacognition as modal cognition. \emph{Conference on Cognitive
Computational Neuroscience}.

\textbf{O'Neill, K.}, Henne, P., Quillien, T., Icard, T., \&
De~Brigard, F. (2025). Norms moderate causal judgments in cases of
double prevention. \emph{Cognitive Science Society}.

Miceli, K., Van Rooy, N., \textbf{O'Neill, K.}, \& De~Brigard,
F. (2024). Causation on a continuum: no normality effects on causal
judgments. \emph{Cognitive Science Society}; \emph{European Society
for Philosophy \& Psychology}.

Murray, S., \textbf{O’Neill, K.}, Bridges, J., Sytsma, J., \& Irving,
Z. (2023). The role of character information in judgments of
blame. \emph{Society for Philosophy and Psychology}.

Fern\'{a}ndez-Miranda, G., \textbf{O’Neill, K.}, Stanley, M., Kushnir,
T., \& De~Brigard, F. (2023). The influence of perceived control on
forgiveness. \emph{Preconference on Justice and Morality, Society
for Personality and Social Psychology}.

Krasich, K., Simmons, C., \textbf{O'Neill, K.}, Giattino, C.M.,
Sinnott-Armstrong, W., De~Brigard, F., Mudrik, L., \& Woldorff,
M.G. (2022). Prestimulus alpha oscillatory activity interacts with
evoked recurrent processing to facilitate conscious visual
perception. \emph{Society for Neuroscience}.

\textbf{O'Neill, K.}, Quillien, T., Henne, P. (2022). A
counterfactual model of causal judgments in double
prevention. \emph{Conference on Cognitive Computational
Neuroscience}.

\textbf{O'Neill, K.}, Krasich, K., Murray, S., Brockmole, J.,
Nuthmann, A., De~Brigard, F. (2022). Fixation duration variability
increases with mind wandering during scene viewing. \emph{Conference
on Cognitive Computational Neuroscience}.

Khoudary, A., \textbf{O’Neill, K.}, Faul, L., Murray, S., Smallman,
R., De~Brigard, F. (2022). Neural differences between internal and
external episodic counterfactual thoughts. \emph{Cognitive
Neuroscience Society Annual Meeting}.

\textbf{O'Neill, K.}, Henne, P., Bello, P., Pearson, J., De~Brigard,
F. (2021). Measuring and modeling confidence in human causal
judgment. \emph{Workshop on Metacognition in the Age of AI:
Challenges and Opportunities, 35th Conference on Neural Information
Processing Systems (NeurIPS 2021), Sydney, Australia}.

\textbf{O'Neill, K.}, Henne, P., Bello, P., Pearson, J., De~Brigard,
F. (2021). Confidence effects on causal judgment. \emph{Psychonomics}.

\textbf{O'Neill, K.}, Henne, P., Bello, P., Pearson, J., De~Brigard,
F. (2021). Degrading causation. \emph{Society for Philosophy and
Psychology Annual Meeting}.

Khoudary, A., Hanna, E., \textbf{O’Neill, K.}, Iyengar, V., Clifford,
S., Cabeza, R., De~Brigard, F., Sinnott-Armstrong, W. (2021). A
functional neuroimaging investigation of moral foundations
theory. \emph{Society for Philosophy and Psychology Annual Meeting};
\emph{2020 meeting of the Cognitive Neuroscience Society}.

Smith, A., \textbf{O'Neill, K.}, Smilek, D., Seli, P. (2019). On the
utility of the dynamic framework of mind
wandering. \emph{Psychonomics}.

Yin, S., \textbf{O'Neill, K.}, Brady, T., De~Brigard, F. (2019). The
effect of category learning on recognition memory: a signal detection
theory analysis. \emph{41st Annual Meeting of the Cognitive Science
Society}.

Lovett, A., Briggs, G., \textbf{O'Neill, K.}, Bello,
P. (2018). Strategic deployment of attention in online causal
judgment: a computational model. \emph{Journal of Vision}, 18(10),
741-741.

Bello, P., Lovett, A., Briggs, G., \textbf{O'Neill, K.} (2018). An
attention-driven model of human causal reasoning. \emph{40th Annual
Meeting of the Cognitive Science Society}.
